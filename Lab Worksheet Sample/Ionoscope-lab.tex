% filepath: h:\OneDrive - West Point\22.05 PH201\1. Lesson Content\26-2 Worksheets\Ionoscope-lab.tex
\documentclass[12pt,addpoints]{exam}

% --- Packages ---
\usepackage[T1]{fontenc}
\usepackage[scaled]{helvet}
\renewcommand{\familydefault}{\sfdefault}
\usepackage{amsmath, amssymb}
\usepackage{siunitx}
\sisetup{per-mode=symbol}
\usepackage{graphicx}
\usepackage{float}
\usepackage{physics}
\usepackage{tikz}
\usepackage{multicol}
\usepackage{enumitem}
\usepackage{geometry}
\usepackage{hyperref}
\usepackage{booktabs}
\geometry{margin=0.75in}

% image path
\graphicspath{{img/}}

\makeatletter
\@ifundefined{WITHANSWERS}{\noprintanswers}{\printanswers}
\makeatother

\renewcommand{\familydefault}{\sfdefault}
\renewcommand{\thequestion}{\alph{question}}
\renewcommand{\thepartno}{\arabic{partno}}

\footer{PH201}{\thepage}{}
\header{IonoScope: Measuring the Ionosphere with GNSS Signals}{}{Name: \underline{\hspace{5cm}}}

\setlength{\parskip}{0.4em}

\begin{document}

\section{IonoScope: Measuring the Ionosphere with GNSS Signals}

\subsection{Learning Objectives}
By the end of this activity you should be able to:
\begin{itemize}
    \item Interpret real GNSS satellite measurement data.
    \item Explain how signal travel time depends on the ionosphere.
    \item Compute and interpret Total Electron Content (TEC).
    \item Perform basic statistical analysis and visualization in Excel.
\end{itemize}

\subsection{Background (from the IonoScope App)}
Smartphones can measure the travel time of radio signals transmitted by navigation satellites.
Signals at different frequencies (L1 and L5) travel at slightly different speeds through the ionosphere due to dispersion.
This delay can be used to estimate the number of free electrons along the path between the satellite and the receiver.

\textbf{Key idea:}
\[
\text{Different frequencies} \Rightarrow \text{different delays} \Rightarrow \text{estimate electron density (TEC)}.
\]

\subsection{Part I -- Understanding the Dataset}

You are given a CSV file exported from IonoScope containing measurements from multiple satellites.

\begin{questions}

\question How many unique satellites appear in the dataset?

\begin{solution}[2cm]
\textbf{12 unique satellites} (using unique pairs of svId + constellation)
\end{solution}

\question Select one satellite with at least 30 measurements. Record the Satellite ID (svId) and GNSS constellation (GPS, Galileo, etc.).

\begin{solution}[2cm]
\textbf{Satellite ID:} 36

\textbf{Constellation:} GALILEO
\end{solution}

\question Go online and determine which country or organization operates this constellation.

\begin{solution}[2cm]
\textbf{European Union / European Space Agency (ESA)}
\end{solution}

\end{questions}

\subsection{Part II -- Signal Delay Verification}

For your selected satellite, choose one measurement (one row).

\begin{questions}

\question Compute the transit time difference:
\[
\Delta t = t_{L5} - t_{L1}
\]
If the transit difference is positive that is a physically meaningful and expected result. If negative it indicates the satellite may have a bias adjustment and shows some of the limitations and challenges of using real-world smartphone data.
\begin{solution}[4cm]
For Galileo-36 (example row):
\begin{align*}
    t_{L1} &= 77{,}998{,}233 \text{ ns} \\
    t_{L5} &= 77{,}998{,}247 \text{ ns} \\
    \Delta t &= t_{L5} - t_{L1} = 77{,}998{,}247 - 77{,}998{,}233 = \boxed{+14 \text{ ns}}
\end{align*}
This positive value is physically meaningful: the L5 signal (lower frequency) travels slower through the ionosphere than L1 (higher frequency), as expected from dispersion.
\end{solution}

\question Compare your result with the provided \texttt{signalDelayNanos} column. Do they match?

\begin{solution}[2cm]
Dataset value: \texttt{signalDelayNanos} $= 14$ ns

$\checkmark$ Match confirmed.
\end{solution}

\question Report the carrier frequencies for L1 and L5 (in MHz).

\begin{solution}[2cm]
\begin{center}
\begin{tabular}{lc}
\toprule
\textbf{Signal} & \textbf{Frequency (MHz)} \\
\midrule
L1 & 1575.42 \\
L5 & 1176.45 \\
\bottomrule
\end{tabular}
\end{center}
\end{solution}

\end{questions}

\subsection{Part III -- TEC Calculation}

The dataset includes a column called \texttt{tecApproximate}.

\begin{questions} 

\question Using the frequency values for L1 and L5 and your computed $\Delta t$, show how TEC is calculated.

\begin{solution}[6cm]
The dispersive delay relation is:
\[
\Delta t = \frac{40.3}{c} \cdot 5\text{TEC} \cdot \left( \frac{1}{f_{L1}^2} - \frac{1}{f_{L1}^2} \right)
\]

Solving for TEC:
\[
\boxed{\text{TEC} = \frac{\Delta t_{L5-L1} \cdot c}{40.3 \left( \dfrac{1}{f_{L5}^2} - \dfrac{1}{f_{L1}^2} \right)}}
\]

Where:
\begin{itemize}
    \item $c = 3.00 \times 10^8$ m/s
    \item Frequencies in Hz (e.g., $f_{L1} = 1.57542 \times 10^9$ Hz)
\end{itemize}
\end{solution}
\clearpage

\question The reported quantity is called ``STEC''. What does this acronym stand for?

\begin{solution}[2cm]
\textbf{STEC = Slant Total Electron Content}

This represents the total number of electrons integrated along the \textit{slanted} path from satellite to receiver. (Vertical TEC would be VTEC.)
\end{solution}

\end{questions}
\subsection{Part IV -- Statistical Analysis in Excel}

Using only your chosen satellite:

\begin{questions} 

\question Compute the mean and standard deviation of TEC. Report your results in TECU.

\textit{Note: 1 TECU $= 10^{16}$ electrons/m$^2$}

\begin{solution}[4cm]
For Galileo-36:

\begin{center}
\begin{tabular}{lcc}
\toprule
\textbf{Quantity} & \textbf{Raw Value} & \textbf{In TECU} \\
\midrule
Mean TEC & $3.26 \times 10^{17}$ electrons/m$^2$ & \textbf{32.6 TECU} \\
Std Dev & $\sim 3 \times 10^{17}$ electrons/m$^2$ & \textbf{$\sim$30 TECU} \\
\bottomrule
\end{tabular}
\end{center}

\textbf{Excel formulas:}
\begin{itemize}
    \item Mean: \texttt{=AVERAGE(range)}
    \item Std Dev: \texttt{=STDEV.S(range)}
\end{itemize}
\end{solution}

\end{questions}

\subsection{Part V -- Time Series Visualization}

\begin{questions} 

\question Create a plot in Excel with \texttt{id} on the x-axis and \texttt{TEC} on the y-axis. Paste or sketch your plot below.

\begin{solution}[5cm]
\textit{[Student pastes Excel plot here]}
\end{solution}

\question Briefly comment on the trend or variability you observe in your plot.

\begin{solution}[3cm]
TEC fluctuates around a roughly constant mean with rapid variations, indicating short-term ionospheric structure and measurement uncertainty. No strong monotonic trend is observed.
\end{solution}

\end{questions}

\clearpage

\subsection{Part VI -- Signal Quality (Carrier-to-Noise)}

\begin{questions} 

\question Compute the average values of \texttt{L1Cn0DbHz} and \texttt{L5Cn0DbHz} for your satellite.

\begin{solution}[3cm]
\begin{center}
\begin{tabular}{lc}
\toprule
\textbf{Signal} & \textbf{Average C/N$_0$ (dB-Hz)} \\
\midrule
L1 & 40.43 \\
L5 & 36.16 \\
\bottomrule
\end{tabular}
\end{center}
\end{solution}

\question Qualitatively describe the signal quality (Excellent / Good / Fair / Poor).

\textit{Rule of thumb: $> 40$ dB-Hz = Excellent; $30$--$40$ dB-Hz = Good; $< 30$ dB-Hz = Poor}

\begin{solution}[2cm]
\begin{itemize}
    \item L1: \textbf{Good--Excellent} (40.43 dB-Hz)
    \item L5: \textbf{Good} (36.16 dB-Hz)
\end{itemize}
\end{solution}

\end{questions}

\subsection{Reflection}

\begin{questions} 

\question In 2--3 sentences: How does this dataset demonstrate that smartphones can be used as scientific instruments?

\begin{solution}[4cm]
This dataset shows that smartphones can measure precise signal timing from multiple GNSS satellites, detecting nanosecond-scale delays caused by ionospheric electrons. By analyzing dual-frequency signals, we can extract quantitative physical information (TEC) about Earth's upper atmosphere---turning a consumer device into a portable ionospheric sensor.
\end{solution}

\end{questions}

\end{document}